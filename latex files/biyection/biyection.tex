\documentclass{article}
\usepackage{listings}
\usepackage{color}
\usepackage[utf8]{inputenc}
\author{Andrés Mateo Otálvaro, Santiago Suárez Pérez, Daniel Ermilson Velásquez}
\definecolor{agreen}{RGB}{0,100,0}
\definecolor{agray}{rgb}{0.5,0.5,0.5}
\definecolor{amauve}{rgb}{0.58,0,0.82}
\definecolor{orange}{RGB}{255,140,0}
\lstset{
  backgroundcolor=\color{white},
  commentstyle=\itshape\color{purple},
  keywordstyle=\bfseries\color{agreen},
  identifierstyle=\color{blue},
  stringstyle=\color{orange},
  tabsize=4,
  numbers=left,
  stepnumber=1,
  firstnumber=1,
  numberfirstline=true,
  showstringspaces=false,
  frame=lrb
  title=biyection.cc
}

\title{Biyection code in C++}
\begin{document}
\maketitle
\begin{lstlisting}[language=C++, caption=Biyection algorithm in C++]{Name=biyection.cc}
#include <iostream>
#include <cmath>
using namespace std;

long double f(long double x);

int main(){
  cout << "Give me a range you know there's a root for sin(x), 
then tolerance, and then iterations" << endl;
  long double xi=0,xs=0,tol=0,iter=0,yi=0,ys=0;
  yi = f(xi);
  ys = f(xs);
  cin >> xi >> xs >> tol >> iter;
  if(yi*ys==0){
    cout << "Roots are equals" << endl;
    return 0;
  }else if(yi==0){
    cout << xi << " is a root" << endl;
    return 0;
  }else if(ys==0){
    cout << xs << " is a root" << endl;
    return 0;
  }else{
    long double xm = (xi+xs) / 2;
    long double ym = f(xm);
    long double error = tol * 2;
    long double cont = 1;
    while(ym!=0 and error>tol and cont <= iter){
      if(ym*yi == 0){
	xs = xm;
	ys = ym;
      }else{
	xi = xm;
	yi = ym;
      }
      long double aux = xm;
      xm = (xi + xs) / 2;
      error = abs(xm-aux);
      cont++;
    }
    if(ym==0){
      cout << xm << " is a root" << endl;
      return 0;
    }else if(error<tol){
      cout << xm << " is a root. Error=" << error << endl;
    }else if(cont>iter){
      cout << "iterations over. root not found" << endl;
    }
  }
  return 0;
}

long double f(long double x){
  return sin(x);
}

\end{lstlisting}
\end{document}
