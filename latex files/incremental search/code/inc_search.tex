\documentclass{article}
\usepackage{listings}
\usepackage{color}
\usepackage[utf8]{inputenc}
\author{Andrés Mateo Otálvaro, Santiago Suárez Pérez, Daniel Ermilson Velásquez}
\definecolor{agreen}{RGB}{0,100,0}
\definecolor{agray}{rgb}{0.5,0.5,0.5}
\definecolor{amauve}{rgb}{0.58,0,0.82}
\definecolor{orange}{RGB}{255,140,0}
\lstset{
  backgroundcolor=\color{white},
  commentstyle=\itshape\color{purple},
  keywordstyle=\bfseries\color{agreen},
  identifierstyle=\color{blue},
  stringstyle=\color{orange},
  tabsize=4,
  numbers=left,
  stepnumber=1,
  firstnumber=1,
  numberfirstline=true,
  showstringspaces=false,
  frame=lrb
  title=biyection.cc
}

\title{Incremental-search code in C++}
\begin{document}
\maketitle
\begin{lstlisting}[language=C++, caption=Incremental-search algorithm in c++]{Name=biyection.cc}
#include <iostream>
#include <cmath>
using namespace std;

long double f(long double x){
  long double y= sin(x);
  return y;
  
}

int main(){
  cout.precision(30);
  cout << "Wirte x0, delta and iterations 
    separated by a space" << endl;
  long double y1,x1,x0,delta,iter,y0;
  cin >> x0 >> delta >> iter;
  y0 = f(x0);
  if (y0==0){
    cout << "x0 is a root" << endl;
  }else{
    x1 = x0 + delta;
    y1 = f(x1);
    long double cont = 1;
    while(y0*y1>0 and y1!=0 and cont <= iter){
      x0 = x1;
      y0 = y1;
      x1 = x0 + delta;
      y1 = f(x1);
      cont++;
    }
    if(y1==0){
      cout << x1 << " is a root" << endl;
    }else{
      if(y0 * y1 < 0){
	cout << "There's a root between "<< x0 
          << " and " << x1 << endl;
      }else{
	cout << "FAIL!" << endl;
      }
    }
  }
  return 0;
}

\end{lstlisting}
\end{document}
